\documentclass[13pt]{amsart}
\usepackage{background}
\usepackage{graphicx}
\usepackage{lipsum}


\usepackage{hyperref}
\usepackage[scaled]{helvet}
\renewcommand\familydefault{\sfdefault} 
\usepackage[T1]{fontenc}
\usepackage[margin=1.37in]{geometry}
\usepackage{verbatim, graphicx, amsmath, mathtools,  enumerate, hyperref, tcolorbox, sourcecodepro}

\definecolor{Red}{cmyk}{0.1, 0.70, 0.65, 0.00, 1.00}
\definecolor{Blue}{cmyk}{0.70, 0.08, 0.08, 0.04, 1.00}
\definecolor{Yellow}{cmyk}{0.0, 0.00, 0.7, 0.00, 0.5}
\definecolor{Green}{cmyk}{0.6, 0.0, 0.6, 0.00, 1.00}
\definecolor{Purple}{cmyk}{0.8, 0.3, 0.3, 0.00, 1.00}
\definecolor{Orange}{cmyk}{0.0, 0.3, 0.7, 0.00, 1.00}


%GET THIS SO IT TAKES IN A COLOR OF THE USER'S CHOOSING.
%GET THIS SO IT TAKES IN PYTHON OUTPUT
%GET THIS SO IT TAKES IN C OUTPUT

%Lean code command
\newcounter{leancounter}
\setcounter{leancounter}{1} 
\newenvironment{lean}[1]{\begin{center}
{\begin{tcolorbox}[width=4.5in,colback={white},title={\begin{center}\texttt{Lean \theleancounter} \addtocounter{leancounter}{1}  \end{center}},colbacktitle=Blue,coltitle=black]  
\ \\   
{\texttt{#1}}
\ \\ 
\end{tcolorbox}  }
\end{center}}

%Lean print command
\newcounter{leancount}
\setcounter{leancount}{1}
\newcommand{\print}[1]{\verbatiminput{\theleancount.txt} \setcounter{leancount}{\theleancount+1} }

%Python code command
\newcounter{pythoncounter}
\setcounter{pythoncounter}{1} 
\newenvironment{python}[1]{\begin{center}
{\begin{tcolorbox}[width=4.5in,colback={white},title={\begin{center}\texttt{Python \thepythoncounter} \addtocounter{pythoncounter}{1}  \end{center}},colbacktitle=Red,coltitle=black]  
\ \\
{\texttt{#1}}
\ \\ 
\end{tcolorbox}  }
\end{center}}

%Lean print command
\newcounter{pythoncount}
\setcounter{pythoncount}{1}
\newcommand{\prynt}[1]{  {\verbatiminput{PY\thepythoncount.txt   }} \setcounter{pythoncount}{\thepythoncount+1} }

%Make cover
\newcommand{\cover}[1]{
\begin{center}
\ \\ \ \\ \ \\
\ \\ \ \\ \ \\
\ \\ \ \\ \ \\
\scalebox{5}{$\mathsf{#1}$} 
\ \\ \ \\ \ \\
\ \\ \ \\ \ \\
$\mathsf{E.\ Dean\ Young}$
\thispagestyle{empty}
\end{center}
\newpage
}
\usepackage{fancyhdr}

\title{Constructing Haar Integral using Insights into Profinite Sets}
\author{E. Dean Young}
\begin{document}


\section{Introduction}

Fix a universe $\texttt{u₀}$. Recall that a topological space $\texttt{TopologicalSpace u₀ : Type}$ consists of:

\begin{enumerate}
\item 
\end{enumerate}

A topological space $\texttt{X : TopologicalSpace u₀}$ is hausdorff if ...\\




A $\texttt{stone\_space : Type}$ is a compact, hausdorff, totally disconnected topological space. 

We can form the category of stone spaces, $\texttt{Stn : Category stone\_space}$, which has $\texttt{stone\_space}$ as its objects. For the purposes of this exposition, a profinite set is an object in $\texttt{Pro.obj FinSet}$. 

There is a categorical equivalence $\texttt{Pro.obj FinSet ≅ Stn}$.

My goal is to put this demonstration of the existence of Haar integral in Lean 4.\\



\begin{enumerate}
\item Definition of an internal groupoid in a category
\item Definition of an internal functor of internal groupoids in a category
\item Definition of $\texttt{IntGrpd C}$
\begin{enumerate}
\item As a Mathlib structure with 16 components:
\begin{enumerate}
\item 
\end{enumerate}
\end{enumerate}
\item Definition of an internal equivalence relation in a category
\item Definition of an internal equivalence relation map of internal equivalence relations in a category
\end{enumerate}

\begin{enumerate}
\item Definition of $\texttt{IntEqvRel C}$
\item Definition of $\texttt{ℝ}$ as a topological space
\item Definition of the Haar integral on an internal groupoid in finite sets
\item Definition of the Haar measure on an internal groupoid in finite sets
\item Definition of the Haar integral on an internal equivalence relation in finite sets
\item Definition of the Haar measure on an internal equivalence relation in finite sets
\item Definition of the functor $\texttt{IntEqvRel (Pro C) ⭢ Top}$
\item Definition of the functor $\texttt{IntEqvRel (Pro C) ⭢ CompactHausdorff}$
\item Definition of the functor $\texttt{(IntEqvRel (Pro FinSet)) ⭢ (Pro (IntEqvRel FinSet))}$
\item Definition of the functor $\texttt{Grp C ⭢ Grp C}$
\item Definition of an ultrafilter u : $\texttt{P(Bool) ⭢ Bool}$
\item "ultrafilter convergence" and compactness
\item "ultrafilter uniqueness" and the Hausdorff condition
\item $\texttt{Functor (Grp (Pro (IntEqvRel FinSet))) (Pro (Grp (IntEqvRel FinSet)))}$
\item Counting measures and counting integrals on $\texttt{Grp (IntEqvRel FinSet)}$
\item The product of internal equivalence relations in 
\begin{enumerate}
\item $\texttt{IntGrp (IntEqvRel FinSet)}$
\item $\texttt{IntEqvRel FinSet}$ consists of ordinary equivalence relations in sets
\end{enumerate}
\end{enumerate}

\begin{enumerate}
\item 
\end{enumerate}

Let $E$ be an internal equivalence relation in finite sets. Write $E.Obj$ for its objects and $E.Hom$ for its morphisms. Note that $(Dom,Cod): E.Hom ⭢ E.Obj \times E.Obj$ is an injective function between finite sets. A function φ : E.Obj ⭢ ℝ such that x $\sim$ y implies φ(x) = φ(y) factors through E/$\sim$ is the same as a certain map of internal equivalence relations in which ℝ gets the equality relation. If we write ℝ${}_{=}$ for the internal equivalence relation on ℝ, then we can think of this as a map in $\texttt{IntEqvRel Set}$. We write [E,ℝ${}_{=}$] for the ℝ-vector space whose elements are functions φ : E.Obj ⭢ ℝ such that x $\sim$ y implies φ(x) = φ(y).
\fi

\section{}



\section{Closed and Internal Equivalence Relations}

\begin{enumerate}[(i)]
\item Definition of a category
\item FinSet
\begin{enumerate}
\item The category FinSet of finite sets 
\end{enumerate}
\item Pro : Cat → Cat (See Johnstone)
\item Top \iffalse
\item Definition of compact
\item Definition of Hausdorff
\item Definition of a cofiltered diagram
\item Definition of a cofiltered limit on objects
\item Definition the category Pro C for a category C
\item The product of profinite sets is profinite and satisfies the universal property of product in profinite sets.
\item Definition of a $\texttt{Pro FinSet}$
\item Definition of a $\texttt{CH}$ the category of compact hausdorff topological spaces, a fully faithful subcategory of topological spaces  \fi
\item Definition of the category of profinite topological spaces
\item Definition of a closed equivalence relation on a profinite set
\item The discrete topological space of a topological space (${}^{dis}$)
\item The Stone-Čech compactification of a topological space (${}^{cmp}$)
\end{enumerate}

    
\begin{definition}
Consider the functor $(-){}^{d}{}^{s} : \texttt{Top} \rightarrow \texttt{Top}$, which is defined as sending a topological space $X$ to the Stone-Čech compactification of discrete space with $\texttt{X.obj}$ as objects, and similarly on maps. If $\texttt{X}$ is a compactum, then there is a map $\texttt{f : X}$${}^{ds}$$\rightarrow$\texttt{X}$ such that $\texttt{X}$$^{}$ 
\end{definition}


{\bf theorem:} Let $X$ be a compact hausdorff topological space. There is a continuous map $Y := X{}^{dis}{}^{cmp} \rightarrow X$ given from the universal property of Stone-Čech compactification applied to the map $X^{dis} \rightarrow Y$. $Y $x ${}_{X} Y$ forms the morphism component of an internal equivalence relation.\\

{\bf theorem:} Let $X$ be a compact hausdorff topological space. There is a continuous map $Y := X{}^{dis}{}^{cmp} \rightarrow X$ given from the universal property of Stone-Čech compactification applied to the map $X^{dis} \rightarrow Y$. $X$ forms the quotient coeq ($\pi_1$, $\pi_2$), where $\pi_1, \pi_2 : Y \times_{X} Y \rightarrow Y$ are the projection maps.\\


\begin{enumerate}[(a)]
\item F and G form a categorical equivalence
\end{enumerate}

See \href{}{Condensed Mathematics} page 17.\\



\section{Haar Integral on Finite Discrete Groups}

\begin{enumerate}
\item 
\end{enumerate}


\section{Haar Integral on Profinite Groups}



\section{Gelfand Duality}

\begin{enumerate}
\item Topology on the Cantor set
\item ℝ in trinary
\item Urysohn's lemma and trinary
\item 
\end{enumerate}

\end{document}





























