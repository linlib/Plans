\documentclass{book}
\usepackage{blindtext, hyperref, verbatim, minted, graphicx, amssymb, textcomp, enumerate, tcolorbox, newunicodechar, textgreek, wasysym, tipa, eso-pic, lipsum, bbold, dsfont}
\usepackage[margin=1.3in]{geometry}
\usepackage{longtable}
\usepackage{lmodern} % Add this line to use the Latin Modern font
\renewcommand\familydefault{\sfdefault}
\usepackage{fontspec}
\usepackage{newunicodechar}

\newenvironment{definition}
{%
    \begin{center}
    \textbf{\large Definition}
    \end{center}
    \begin{quote}
}
{%
    \end{quote}
}

\newunicodechar{×}{$\times$}
\newunicodechar{→}{$\rightarrow$}
\newunicodechar{⟨}{$\langle$}
\newunicodechar{⟩}{$\rangle$}
\newunicodechar{↦}{$\mapsto$}
\newunicodechar{∧}{$\wedge$}
\newunicodechar{∨}{$\vee$}
\newunicodechar{∃}{$\exists$}
\newunicodechar{∀}{$\forall$}
\newunicodechar{¬}{$\neg$}

\newunicodechar{ᵃ}{${}^{\texttt{a}}$}
\newunicodechar{ᵇ}{${}^{\texttt{b}}$}
\newunicodechar{ᶜ}{${}^{\texttt{c}}$}
\newunicodechar{ᵈ}{${}^{\texttt{d}}$}
\newunicodechar{ᵉ}{${}^{\texttt{e}}$}
\newunicodechar{ᶠ}{${}^{\texttt{f}}$}
\newunicodechar{ᵍ}{${}^{\texttt{g}}$}
\newunicodechar{ʰ}{${}^{\texttt{h}}$}
\newunicodechar{ⁱ}{${}^{\texttt{i}}$}
\newunicodechar{ʲ}{${}^{\texttt{j}}$}
\newunicodechar{ᵏ}{${}^{\texttt{k}}$}
\newunicodechar{ˡ}{${}^{\texttt{l}}$}
\newunicodechar{ᵐ}{${}^{\texttt{m}}$}
\newunicodechar{ⁿ}{${}^{\texttt{n}}$}
\newunicodechar{ᵒ}{${}^{\texttt{o}}$}
\newunicodechar{ᵖ}{${}^{\texttt{p}}$}
\newunicodechar{ʳ}{${}^{\texttt{r}}$}
\newunicodechar{ˢ}{${}^{\texttt{s}}$}
\newunicodechar{ᵗ}{${}^{\texttt{t}}$}
\newunicodechar{ᵘ}{${}^{\texttt{u}}$}
\newunicodechar{ᵛ}{${}^{\texttt{v}}$}
\newunicodechar{ʷ}{${}^{\texttt{w}}$}
\newunicodechar{ˣ}{${}^{\texttt{x}}$}
\newunicodechar{ʸ}{${}^{\texttt{y}}$}
\newunicodechar{ᶻ}{${}^{\texttt{z}}$}

\newunicodechar{⁰}{${}^{\texttt{0}}$}
\newunicodechar{¹}{${}^{\texttt{1}}$}
\newunicodechar{²}{${}^{\texttt{2}}$}
\newunicodechar{³}{${}^{\texttt{3}}$}
\newunicodechar{⁴}{${}^{\texttt{4}}$}
\newunicodechar{⁵}{${}^{\texttt{5}}$}
\newunicodechar{⁶}{${}^{\texttt{6}}$}
\newunicodechar{⁷}{${}^{\texttt{7}}$}
\newunicodechar{⁸}{${}^{\texttt{8}}$}
\newunicodechar{⁹}{${}^{\texttt{9}}$}

\newunicodechar{⁻}{${}^{\texttt{-}}$}

\newunicodechar{ᵒ}{${}^{\texttt{o}}$}
\newunicodechar{ᵖ}{${}^{\texttt{p}}$}

\newunicodechar{⁻}{${}^{\texttt{-}}$}
\newunicodechar{¹}{${}^{\texttt{1}}$}

\newunicodechar{₀}{${}_{\texttt{0}}$}
\newunicodechar{₁}{${}_{\texttt{1}}$}
\newunicodechar{₂}{${}_{\texttt{2}}$}
\newunicodechar{₃}{${}_{\texttt{3}}$}
\newunicodechar{₄}{${}_{\texttt{4}}$}
\newunicodechar{₅}{${}_{\texttt{5}}$}
\newunicodechar{₆}{${}_{\texttt{6}}$}
\newunicodechar{₇}{${}_{\texttt{7}}$}
\newunicodechar{₈}{${}_{\texttt{8}}$}
\newunicodechar{₉}{${}_{\texttt{9}}$}

\newunicodechar{𝚫}{$\Delta$}

\newunicodechar{ʃ}{$\int$}
\newunicodechar{⊗}{$\otimes$}


\newunicodechar{𝔸}{$\mathbb{A}$}
\newunicodechar{𝔹}{$\mathbb{B}$}
\newunicodechar{ℂ}{$\mathbb{C}$}
%\newunicodechar{}{$\mathbb{D}$}
%\newunicodechar{}{$\mathbb{E}$}
\newunicodechar{𝔽}{$\mathbb{F}$}
\newunicodechar{𝔾}{$\mathbb{G}$}
%\newunicodechar{}{$\mathbb{H}$}
%\newunicodechar{}{$\mathbb{I}$}
%\newunicodechar{}{$\mathbb{J}$}
%\newunicodechar{}{$\mathbb{K}$}
%\newunicodechar{}{$\mathbb{L}$}
%\newunicodechar{}{$\mathbb{M}$}
\newunicodechar{ℕ}{$\mathbb{N}$}
%\newunicodechar{}{$\mathbb{O}$}
\newunicodechar{ℙ}{$\mathbb{P}$}
%\newunicodechar{}{$\mathbb{Q}$}
\newunicodechar{ℝ}{$\mathbb{R}$}
\newunicodechar{𝕊}{$\mathbb{S}$}
%\newunicodechar{}{$\mathbb{T}$}
%\newunicodechar{}{$\mathbb{U}$}
%\newunicodechar{}{$\mathbb{V}$}
%\newunicodechar{}{$\mathbb{W}$}
%\newunicodechar{}{$\mathbb{X}$}
%\newunicodechar{}{$\mathbb{Y}$}
\newunicodechar{ℤ}{$\mathbb{Z}$}
\newunicodechar{𝕒}{$\mathbb{a}$}
\newunicodechar{𝕓}{$\mathbb{b}$}
%\newunicodechar{}{$\mathbb{c}$}
\newunicodechar{𝕕}{$\mathbb{d}$}
\newunicodechar{𝕖}{$\mathbb{e}$}
\newunicodechar{𝕗}{$\mathbb{f}$}
%\newunicodechar{}{$\mathbb{g}$}
%\newunicodechar{}{$\mathbb{h}$}
%\newunicodechar{}{$\mathbb{i}$}
\newunicodechar{𝕛}{$\mathbb{j}$}
%\newunicodechar{}{$\mathbb{k}$}
%\newunicodechar{}{$\mathbb{l}$}
%\newunicodechar{}{$\mathbb{m}$}
%\newunicodechar{}{$\mathbb{n}$}
\newunicodechar{𝕠}{$\mathbb{o}$}
\newunicodechar{𝕡}{$\mathbb{p}$}
%\newunicodechar{}{$\mathbb{q}$}
\newunicodechar{𝕣}{$\mathbb{r}$}
%\newunicodechar{}{$\mathbb{s}$}
\newunicodechar{𝕥}{$\mathbb{t}$}
%\newunicodechar{}{$\mathbb{u}$}
%\newunicodechar{}{$\mathbb{v}$}
%\newunicodechar{}{$\mathbb{w}$}
%\newunicodechar{}{$\mathbb{x}$}
%\newunicodechar{}{$\mathbb{y}$}
%\newunicodechar{}{$\mathbb{z}$}

\usepackage{eucal}
\usepackage{euscript}

\newunicodechar{𝔄}{$\mathfrak{A}$}
\newunicodechar{𝔅}{$\mathfrak{B}$}
\newunicodechar{ℭ}{$\mathfrak{C}$}
\newunicodechar{𝔇}{$\mathfrak{D}$}
\newunicodechar{𝔈}{$\mathfrak{E}$}
\newunicodechar{𝔉}{$\mathfrak{F}$}
\newunicodechar{𝔊}{$\mathfrak{G}$}
\newunicodechar{ℌ}{$\mathfrak{H}$}
\newunicodechar{ℑ}{$\mathfrak{I}$}
\newunicodechar{𝔍}{$\mathfrak{J}$}
\newunicodechar{𝔎}{$\mathfrak{K}$}
\newunicodechar{𝔏}{$\mathfrak{L}$}
\newunicodechar{𝔐}{$\mathfrak{M}$}
\newunicodechar{𝔑}{$\mathfrak{N}$}
\newunicodechar{𝔒}{$\mathfrak{O}$}
\newunicodechar{𝔓}{$\mathfrak{P}$}
\newunicodechar{𝔔}{$\mathfrak{Q}$}
\newunicodechar{ℜ}{$\mathfrak{R}$}
\newunicodechar{𝔖}{$\mathfrak{S}$}
\newunicodechar{𝔗}{$\mathfrak{T}$}
\newunicodechar{𝔘}{$\mathfrak{U}$}
\newunicodechar{𝔙}{$\mathfrak{V}$}
\newunicodechar{𝔚}{$\mathfrak{W}$}
\newunicodechar{𝔛}{$\mathfrak{X}$}
\newunicodechar{𝔜}{$\mathfrak{Y}$}
\newunicodechar{ℨ}{$\mathfrak{Z}$}

\newunicodechar{𝔞}{$\mathfrak{a}$}
\newunicodechar{𝔟}{$\mathfrak b$}
\newunicodechar{𝔠}{$\mathfrak{c}$}
\newunicodechar{𝔡}{$\mathfrak{d}$}
\newunicodechar{𝔢}{$\mathfrak{e}$}
\newunicodechar{𝔣}{$\mathfrak{f}$}
\newunicodechar{𝔤}{$\mathfrak{g}$}
\newunicodechar{𝔥}{$\mathfrak{h}$}
\newunicodechar{𝔦}{$\mathfrak{i}$}
\newunicodechar{𝔧}{$\mathfrak{j}$}
\newunicodechar{𝔨}{$\mathfrak{k}$}
\newunicodechar{𝔩}{$\mathfrak{l}$}
\newunicodechar{𝔪}{$\mathfrak{m}$}
\newunicodechar{𝔫}{$\mathfrak{n}$}
\newunicodechar{𝔬}{$\mathfrak{o}$}
\newunicodechar{𝔭}{$\mathfrak{p}$}
\newunicodechar{𝔮}{$\mathfrak{q}$}
\newunicodechar{𝔯}{$\mathfrak{r}$}
\newunicodechar{𝔰}{$\mathfrak{s}$}
\newunicodechar{𝔱}{$\mathfrak{t}$}
\newunicodechar{𝔲}{$\mathfrak{u}$}
\newunicodechar{𝔳}{$\mathfrak{v}$}
\newunicodechar{𝔴}{$\mathfrak{w}$}
\newunicodechar{𝔵}{$\mathfrak{x}$}
\newunicodechar{𝔶}{$\mathfrak{y}$}
\newunicodechar{𝔷}{$\mathfrak{z}$}



\newunicodechar{𝐂}{${\bf{C}}$}
\newunicodechar{𝐈}{${\bf{I}}$}
\newunicodechar{𝐒}{${\bf{S}}$}

\newunicodechar{𝐚}{${\bf{a}}$}
\newunicodechar{𝐞}{${\bf{e}}$}
\newunicodechar{𝐟}{${\bf{f}}$}
\newunicodechar{𝐧}{${\bf{n}}$}
\newunicodechar{𝐭}{${\bf{t}}$}

\newunicodechar{⊣}{\ensuremath{\dashv}}
\newunicodechar{ॱ}{${}_{\cdot}$}
\newunicodechar{𛲔}{${}^{\cdot}$}
\newunicodechar{⇄}{$\rightleftarrows$}
\newunicodechar{⇆}{$\leftrightarrows$}

\newunicodechar{ꜝ}{$\raisebox{1ex}{\scalebox{0.5}{\texttt{!}}}$}
\newunicodechar{ꜞ}{$\raisebox{1ex}{\scalebox{0.5}{\texttt{¡}}}$}

\newunicodechar{𝟙}{$\mathbb{1}$}
\newunicodechar{∘}{$\circ$}

\newunicodechar{𝟏}{${\bold{1}}$}
\newunicodechar{⭢}{$\longrightarrow$}
\newunicodechar{•}{${\bullet}$}
\newunicodechar{∙}{${\bullet}$}

\newunicodechar{⇉}{$\rightrightarrows$}
\newunicodechar{よ}{$\includegraphics[width=0.27cm,height=0.27cm]{yon.png}$}

\newunicodechar{⊥}{$\bot$}

\newunicodechar{∼}{$\sim$}
\newunicodechar{≃}{$\simeq$}
\newunicodechar{≅}{$\cong$}
\newunicodechar{∞}{$\infty$}

\newunicodechar{α}{$\alpha$}
\newunicodechar{β}{$\beta$}
\newunicodechar{γ}{$\gamma$}
\newunicodechar{δ}{$\delta$}
\newunicodechar{ε}{$\epsilon$}
\newunicodechar{η}{$\eta$}
\newunicodechar{ζ}{$\zeta$}
\newunicodechar{θ}{$\theta$}
\newunicodechar{ι}{$\iota$}
\newunicodechar{μ}{$\mu$}
\newunicodechar{κ}{$\kappa$}
\newunicodechar{λ}{$\lambda$}
\newunicodechar{ρ}{$\rho$}
\newunicodechar{π}{$\pi$}
\newunicodechar{σ}{$\sigma$}
\newunicodechar{τ}{$\tau$}
\newunicodechar{υ}{$\upsilon$}
\newunicodechar{φ}{$\phi$}
\newunicodechar{ψ}{$\psi$}
\newunicodechar{χ}{$\chi$}
\newunicodechar{ω}{$\omega$}



\makeatletter
\newcommand*{\shifttext}[2]{\settowidth{\@tempdima}{#2}\makebox[\@tempdima]{\hspace*{#1}#2}}
\makeatother
\definecolor{Red}{cmyk}{0.1, 0.70, 0.65, 0.00, 1.00}
\definecolor{Blue}{cmyk}{0.9, 0.2, 0.2, 0.00, 1.00}
\definecolor{Yellow}{cmyk}{0.0, 0.00, 0.7, 0.00, 0.5}
\definecolor{Green}{cmyk}{0.6, 0.0, 0.6, 0.00, 1.00}
\definecolor{Purple}{cmyk}{0.8, 0.3, 0.3, 0.00, 1.00}
\definecolor{Orange}{cmyk}{0.0, 0.3, 0.7, 0.00, 1.00}
\definecolor{Grey}{cmyk}{0.13, 0.13, 0.13, 0.00, 1.00}
\newcounter{definitioncounter}
\setcounter{definitioncounter}{1}
\newcounter{theoremcounter}
\setcounter{theoremcounter}{1}
\newcounter{printcounter}
\setcounter{printcounter}{1}
\newcounter{examplecounter}
\setcounter{examplecounter}{1}
\newcounter{ccounter}
\setcounter{ccounter}{1}
\newcounter{pcounter}
\setcounter{pcounter}{1}
\newcounter{lcounter}
\setcounter{lcounter}{1}
\newcounter{sectioncount}
\newcounter{subsectioncount}
\setcounter{sectioncount}{1}
\renewcommand{\section}[1]{\newpage\ \\ \ \\ \begin{center} \scalebox{1.5}{\texttt{\thesectioncount . #1}} \stepcounter{sectioncount} \setcounter{subsectioncount}{1} \end{center} \begin{center} \ \\ \ \\ \thispagestyle{empty} \end{center}}
\renewcommand{\subsection}[1]{\texttt{\thesubsectioncount . #1} \stepcounter{subsectioncount}}
\renewcommand{\backslash}{\reflectbox{\texttt{/}}}



\renewcommand{\chapter}[1]{
\newpage
{
\Huge 
\begin{center}
\ \\
\ \\
\thispagestyle{empty}
\texttt{#1}
\end{center}}
\ \\
\ \\
}

\begin{document}

In the document below I state my interests and objectives concerning mathematics in the Lean 4 computer proof assistant. Specifically, I want to develop many of these ideas into PRs (pull requests) for Mathlib 4. Over the next four year period, I would like to complete what I below describe as stages I and II of a plan. 

These plans are detailed in ...

\section{\scalebox{0.6}{Stage I: goals concerning homotopy and stable homotopy}}

Over the course of the next many years, I hope to develop the repositories at $\texttt{github.com/linlib}$, particularly and first of all $\texttt{ThreeWhiteheadTheoremsandThreePuppeSequences}$, into PRs for Mathlib 4. This is an ambitious goal, and it features objects which are of central importance in mathematics. The most significant and mentioned obstacle in this regard is the need for an API (application programming interface). Considerations in the interface associated to a theory of ∞-categories in Lean are especially important given the central role of the material. However, while the proof assistants Agda and Coq have developed extensive libraries for these concepts, the approaches to homotopy in Lean 3 and 4 remain largely personal projects. In addition, there are many desirable features that a PR for these concepts could have. Here are some other considerations:

\begin{enumerate}[(a)]
\item There are many models for ∞-categories, each of which satisfies the four goals expressed in the repository $\texttt{ThreeWhiteheadTheoremandThreePuppeSequences}$ for ∞-categories (the first of the twelve mentioned goals). 
\item It is best to start small, and to get one part correct at a time. Large PRs are hard to check and more prone to error.
\item Many mathematicans are more interested in CW-complexes and point-set approaches to ∞-groupoids than the full view presented in $\texttt{ThreeWhiteheadTheoremandThreePuppeSequences}$. It is desirable to have an approach which integrates modern and classical approaches. Further, the approaches which use simplicial sets (see $\textt{SSet}$) and point-set topology benefit from being accessible and familiar.
\item It is important to ensure that code is usable for future projects, especially those which will add important API features.
\end{enumerate}

\iffalse
https://leanprover.zulipchat.com/#narrow/stream/116395-maths/topic/2-categories
\fi


\iffalse
The concept of a model . 
e like quite a large API design project.
\fi

\iffalse
On the other hand, a more careful inspection of the existing approaches reveals a contrasting problem, namely that... divisiveness has resulted in .

\begin{enumerate}
\item 
\end{enumerate}
\fi

One discussion on the Lean 4 Zulip from 2020 contained a similar observation, in which the prospects of a "plain formalization" are mentioned:

https://leanprover.zulipchat.com/#narrow/stream/116395-maths/topic/quasicategories

\iffalse
https://leanprover.zulipchat.com/#narrow/stream/113489-new-members/topic/New.20member
https://leanprover.zulipchat.com/#narrow/stream/116395-maths/topic/2-categories
https://leanprover.zulipchat.com/#narrow/stream/113488-general/topic/Lean.20vs.20Coq
https://leanprover.zulipchat.com/#narrow/stream/116395-maths/topic/Three.20Whitehead.20theorems
https://leanprover.zulipchat.com/#narrow/stream/335062-homology/topic/Spectral.20sequences
https://leanprover.zulipchat.com/#narrow/stream/267928-condensed-mathematics/topic/The.20elephant.20in.20the.20room
https://leanprover.zulipchat.com/#narrow/stream/113489-new-members/topic/New.20member
https://leanprover.zulipchat.com/#narrow/stream/113489-new-members/topic/New.20member
https://leanprover.zulipchat.com/#narrow/stream/116395-maths/topic/Localization.20of.20categories
https://leanprover.zulipchat.com/#narrow/stream/236446-Type-theory/topic/How.20to.20get.20HoTT.20people.20into.20Lean
https://leanprover.zulipchat.com/#narrow/stream/113488-general/topic/seminars
https://leanprover.zulipchat.com/#narrow/stream/113488-general/topic/Lean.20vs.20Coq
https://leanprover.zulipchat.com/#narrow/stream/113488-general/topic/Lean.20vs.20Coq
https://leanprover.zulipchat.com/#narrow/stream/116395-maths/topic/With.20Category.20Theory.2C.20Mathematics.20Escapes.20From.20Equality
https://leanprover.zulipchat.com/#narrow/stream/113488-general/topic/Infinity.20Hotel
https://leanprover.zulipchat.com/#narrow/stream/144837-PR-reviews/topic/.231160.20fundamental.20groupoid
https://leanprover.zulipchat.com/#narrow/stream/116395-maths/topic/condensed.20mathematics
https://leanprover.zulipchat.com/#narrow/stream/116395-maths/topic/condensed.20mathematics
https://leanprover.zulipchat.com/#narrow/stream/116395-maths/topic/With.20Category.20Theory.2C.20Mathematics.20Escapes.20From.20Equality
https://leanprover.zulipchat.com/#narrow/stream/113488-general/topic/how.20zfc.20works
https://leanprover.zulipchat.com/#narrow/stream/236446-Type-theory/topic/How.20to.20get.20HoTT.20people.20into.20Lean
https://leanprover.zulipchat.com/#narrow/stream/113488-general/topic/Intuitionistic.20Type.20Theory.20Proofs



Ramkumar Ramachandra
Joel
\fi

\iffalse
Needless to say, there are extensive differences between approaches to the study of ∞-categories and ∞-groupoids based in HoTT (homotopy type theory) and what some have called "plain foramlization". Such a plain formalization could make use of ETCC (elementary theory of the category of categories), which would involve defining the following structures along with corresponding results involving replacement for dependent types:

\begin{enumerate}
\item Category
\item Functor 
\item Natural$\_$Transformation
\item Equation (between natural transformations)
\end{enumerate}
\fi

The Lean community is especially suited towards such a `plain formalization' approach. \\

The category of quasicategories and the category of Kan complexes are both essentially determined from the pre-existing material within Mathlib 4. Specifically, we below list those structures involved in the definition of quasicategories and Kan complexes:

\begin{enumerate}
\item Horns
\item Simplices
\item The horn inclusions
\end{enumerate}

Given the four important considerations (a)-(d) at the outset of the chapter, we would like to consider a pull request based both on the ultimate and far-reaching goals of the first repository. After explaining the content and main features of the pull request, we proceed to explain how we address the important concerns made at the outset.

\begin{enumerate}
\item 
\end{enumerate}

API 




\iffalse
It is also important to explicitly name the category of quasicategories and the category of 
-
-
\fi


The twelve goals in the repository $\texttt{ThreeWhiteheadTheoremsandThreePuppeSequences}$ have allowed for an approach which is motivated by results and which is not particular as to the choice of model. \\

\iffalse
{\bf Why Model} ...
\fi

\iffalse
https://leanprover.zulipchat.com/#narrow/stream/116395-maths/topic/2-categories
\fi

In stage one, we would like to detail twelve major theorems:

\begin{enumerate}
\item The Whitehead theorem
\item The Puppe sequence and its exactness
\item An analogue of the Whitehead theorem for ∞-groupoids featuring an interval object along with two different structures from internal category theory (see Janelidze and Bourceux chapter 7).
\item An analogue of the Puppe sequence for ∞-groupoids featuring an interval object along with two different structures from internal category theory (see Janelidze and Bourceux chapter 7).
\end{enumerate}

{\bf STAGE I (Twenty Four Goals):}
\begin{enumerate}
\item (Three Whitehead Theorems and Three Puppe Sequences, 2024)
\begin{enumerate}
\item See the goals listed in the pdf file under the $\texttt{Three Whitehead Theorems and Three Puppe Sequences}$ repository.
\item Term length, typeclasses, and inductive types
\end{enumerate}
\item (Three Freudenthal Suspension Theorems and Three Long Exact Sequences, 2025)
\begin{enumerate}
\item See the goals listed in the pdf file under the $\texttt{Three Freudenthal Suspension Theorems and Three Long Exact Sequences}$ repository 
\item Term length, typeclasses, and inductive types
\end{enumerate}
\end{enumerate}



\section{\scalebox{0.6}{Stage II: Goals concerning geometric maps in topos theory and stable homotopy}}

In $\textit{Foundations for Computable Topology}$, Paul Taylor developed a synthesis of the logic of topological frames, exposing a duality between the topological concepts of local homeomorphism and proper map, and revealing a new condition (overtness and the analogous condition for maps). All topological spaces are overt, but not all locales.\\

These ideas have also been developed by Martin Escardo.\\

One of the most interesting features of this synthesis (abstract stone duality) is that it allows for compact intersections of open sets and compact unions of closed sets, a dual concept to (discrete overt) unions of open sets and (discrete) intersections of closed sets.\\

A frame can be endowed with a topology using the compact-open topology on the internal hom [X,2]. This topology is the same as the Scott topology.\\

In one ensuing line of inquiry, the preservation of cofiltered limits and filtered colimits is related to the continuity of a homomorphism of frames.\\

This project (Stage II) takes as its main interest the logical patterns of the work of Taylor and Escardo. Our goal is to make many of the same proofs work for higher topos theory and higher algebra, in which the open/closed classifier gets replaced by an internal universe object.\\ 

The analogues fall within the scope of topos theory and higher algebra.\\

{\bf STAGE II (TOPOLOGY):}
\begin{enumerate}
\item (Topology Project I, 2026)
\begin{enumerate}
\item The construction of the following universe objects:
\begin{enumerate}
\item ∞${}\_$(∞-Cat) : ∞-Cat.α
\item ∞${}\_$(∞-Grpd) : ∞-Cat.α
\item ∞${}\_$(∞-Grpd₀) : ∞-Cat.α
\item D(∞${}\_$(∞-Cat)) : ∞-Cat.α
\item D(∞${}\_$(∞-Grpd)) : ∞-Cat.α
\item D(∞${}\_$(∞-Grpd₀)) : ∞-Cat.α
\end{enumerate}
\item The construction of the following:
\begin{enumerate}
\item ∞${}\_$(∞-Cat)⁄C : ∞-Cat.α
\item ∞${}\_$(∞-Grpd)⁄G : ∞-Cat.α
\item ∞${}\_$(∞-Grpd₀)⁄G₀ : ∞-Cat.α
\item D(∞${}\_$(∞-Cat)⁄C) : ∞-Cat.α
\item D(∞${}\_$(∞-Grpd)⁄G) : ∞-Cat.α
\item D(∞${}\_$(∞-Grpd₀)⁄G₀) : ∞-Cat.α
\end{enumerate}
\item The theory of pointed Kan extensions from the perspective of directed homotopy pullback of ⊥ : * ⭢ ∞${}\_$(∞-Cat)... or perhaps some kind of directed derived category instead. See our "\texttt{pullback}$\_$\texttt{system}" structure.
\item The relationship between the projection formulas and the condition of being an open or closed map in frame-local theory. See \href{https://ncatlab.org/nlab/show/closed+morphism}{here} and \href{https://ncatlab.org/nlab/show/open+morphism}{here}.
\item Defining etale unions and proper intersections in the style of Martin Escardo and Paul Taylor.
\item Proper intersections of closed are closed and etale unions of opens are open.
\item Proper iff universally closed, Étale iff universally open.
\item Proper iff cofiltered limit in the "category of actions", Étale iff filtered colimit in the "category of actions". In the case of frame-local theory, this is a certain structure to do with a frame (a particular join lattice), but viewed in the category of complete partial orders as opposed to frames and locales.
\item Tychonoff's theorem: 
\end{enumerate}
\item (Topology Project II, 2027):
\begin{enumerate}
\item The construction of the following universe objects:
\begin{enumerate}
\item ∞${}\_$(Directoid) : ∞-Cat.α
\item ∞${}\_$(Spectroid) : ∞-Cat.α
\item ∞${}\_$(Spectra) : ∞-Cat.α
\item D(∞${}\_$(Directoid)) : ∞-Cat.α
\item D(∞${}\_$(Spectroid)) : ∞-Cat.α
\item D(∞${}\_$(Spectra)) : ∞-Cat.α
\end{enumerate}
\item The construction of the following universe objects:
\begin{enumerate}
\item Mon D(∞${}\_$(Directoid)) : ∞-Cat.α
\item Mon D(∞${}\_$(Spectroid)) : ∞-Cat.α
\item Mon D(∞${}\_$(Spectra)) : ∞-Cat.α
\item Com D(∞${}\_$(Directoid)) : ∞-Cat.α
\item Com D(∞${}\_$(Spectroid)) : ∞-Cat.α
\item Com D(∞${}\_$(Spectra)) : ∞-Cat.α
\end{enumerate}
\item The construction of the following:
\begin{enumerate}
\item Act D : ∞-Cat.α 
\item Act S : ∞-Cat.α
\item Act S : ∞-Cat.α
\end{enumerate}
\item Ramification 
\item The closed and open conditions and their relationship with the projection formulas.
\item Universally closed intersections of closed maps are closed and universally open etale unions of opens are open.
\item Showing that proper is equivalent to being universally closed with universally closed diagonal, and that Étale is equivalent to being universally open with universally open diagonal. This much can be shown first for the case of partial orders, and then for the case of an unstraightening system.
\item Proper iff cofiltered limit in the category of actions, Étale iff filtered colimit in the category of actions.
\item Tychonoff's theorem.
\item What is "Conglomeration Index"?
\item Ramified extensions, Unseparated extensions.
\end{enumerate}
\end{enumerate}


\section{STAGE III: Number theory and beyond}

It will not be for many years that the ideas in this section come to fruition and take their place on the computer. The two stages before this are quite extensive.\\

\subsection{Connecting the Dots with the Rings in Mathlib 4}

In this section, we will define the following rings:

\begin{enumerate}
\item ℚ
\item ℝ, ℝˢᵉᵖ, ℂ
\item ℤₚ
\item ℤ̂
\item ℤ̂ ≅ Π ℤₚ
\item ℚₚ, ℚₚˢᵉᵖ, ℂₚ
\item 𝔸ᶠⁱⁿ
\item 𝔸
\end{enumerate}

Each of these will be defined first as E∞-rings using the models in the repository of  $\texttt{ThreeStableHomotopyCategoriesandThreeLongExactSequences}$. We will also use the definition of unramified extensions established in the fourth repository,  $\texttt{GeometricMapsinStableHomotopyTheory}$. We then relate these E∞-rings to their corresponding rings via structure preserving bijections.\\



\subsection{Spectral Sequences and Exact Couples}

\begin{enumerate}
\item 
\end{enumerate}

\subsection{Derivations and Flat Connections}





\begin{enumerate}
\item Derivations 
\begin{enumerate}
\item Derivations as elements of the ℝ-linear dual of ...
\item H₀ ([Ω∞ X,ℝ])
\end{enumerate}
\item Connections
\begin{enumerate}
\item Flat connections as elements of the ℝ-linear dual of $\texttt{Ω\_(Spectra)C}$, where C is the cohomology content (an E∞-ring)
\item H₀ ([ω∞ X V, ℝ])
\end{enumerate}
\end{enumerate}

\subsection{Trace and Determinant}

\begin{enumerate}
\item The main properties of 𝕋𝕣 := Σ${}_{i}$ $(-1)^{i}$ * Tr${}^{(-1)^{i}}$(Φ)
\item The main properties of 𝔻𝕖𝕥 := Π${}_i$ Det${}^{(-1)^{i}}$(Φ)
\end{enumerate}

\subsection{Borel-Moore Homology and Cohomology with Compact Support}

Borel-Moore homology and cohomology with compact support:\\

\begin{enumerate}
\item Homology, p𛲔 pॱ 𝟙
\item Cohomology Cmp(p)𛲔 pॱ 𝟙
\item Borel-Moore Cohomology Cmp(p)ॱ Cmp(p)𛲔
\item Cohomology with compact support pॱ Cmp(p)𛲔
\end{enumerate}

https://homepages.warwick.ac.uk/staff/Martin.Gallauer/docs/m6ff.pdf

I would eventually like to follow through on several analogous  "étalification" based cohomology theories:

\begin{enumerate}
\item Homology, p𛲔 pॱ 𝟙
\item ??? Ét(p)𛲔 pॱ 𝟙
\item ??? Ét(p)ॱ Ét(p)𛲔
\item ??? pॱ Ét(p)𛲔
\end{enumerate}

I don't know of any "étalification", except one which on fibers takes the underlying discrete topology. That much shows up in the stone-čech compactification of the underlying set of a compact hausdorff topological space X, which has a canonical map into X. The condition on a topological space that it have an open diagonal is equivalent to discrete. The condition on a universally open map f : X ⭢ Y that Δ f f : X ⭢ X ×_(Y) X be open or universally open is related to being étale. 

\subsection{Poincare Duality}

\begin{enumerate}
\item The Cayley-Hamilton Theorem for objects with Poincare duality
\item Pullback and cup product
\end{enumerate}

\subsection{The Kunneth Theorem}

\begin{enumerate}
\item length
\item Tor and Ext
\item The significance of local rings.
\end{enumerate}

\subsection{Intersection Theory}

In this section I would like to establish a few core results concerning intersections. By the date of the implementation of stage III, we will have finished the following repositories:

\begin{enumerate}[(a)]
\item ThreeWhiteheadTheoremsandThreePuppeSequences
\item ThreeStableHomotopyCategoriesandThreeLongExactSequences
\item GeometricMapsinHigherToposTheory
\item GeometricMapsinStableHomotopyTheory
\end{enumerate}

We will then be in a place to consider an intersection theory using the smash product, ⊗. We also develop two generalizations of the smash product for spectroids and directoids.\\

\begin{enumerate}
\item Well behaved cycle maps between pullback and cohomology classes.
\item We would like to mention how these results relate to Bézout's theorem.
\item https://en.wikipedia.org/wiki/Intersection_theory
\end{enumerate}

\subsection{The Lefschetz Fixed Point Theorem}

\begin{enumerate}
\item The Lefschetz Fixed Point Theorem follows from results in intersection theory concerning the cycle map, the Kunneth theorem, and Poincare duality.
\item 
\end{enumerate}



\subsection{The Cayley-Hamilton Theorem}

\begin{enumerate}
\item det(Φ- λ I) = (Φ - λ I) • adj(Φ- λ I).
\item End(V ⊗ k[t]) ≅ End(V)  ⊗ k[t] for V a free k-action.
\item The theorem is first shown for the case of a finite free module, and then closed under quotients.
\end{enumerate}

\subsection{Exp and Log}

\begin{enumerate}
\item ℚ contains the coefficients in the power series of both exp and log
\item Cohomology with rational coefficients
\item E.obj ℤ × E.obj ℤ
\item ℤ ⟳ (E.obj ℤ)
\item 
\item (E.obj ℤ) × (E.obj ℤ) ⭢ (B.obj ℤ) × (E.obj ℤ)
\item 
\item exp : ℂ ⭢ ℂˣ
\item log : π⃡₀.obj (P⃡.obj ℂˣ) ⭢ ℂ
\item 1/(1-tz)
\item Under a wide variety of circumstances, the product over n of det( H₀(Ωⁿ(Φ)) ) to the power (-1)ⁿ is the unique invariant with a multiplication rule for exact sequences.
\item adj( 1 - t * Φ ) * ( 1 - t * Φ ) = det( 1 - t * Φ )
\item log( 1/(1 - tΦ) )
\item log( 1/(1 - t * Φ) ) = - log( 1-t * Φ ) 
\item log(1 - t * Φ) = - ʃ (n : Nat)  Pow n (t * Φ)
\item If \(\Phi\) is a complex number within the unit disc (i.e., \(|\Phi| < 1\)), and you are looking to express \(\log(1 - t\Phi)\) as a sum over the natural numbers \(ℕ\), then we can consider using the Taylor series expansion of the logarithmic function.

The Taylor series expansion of \(\log(1 - x)\) around \(x = 0\) is given by:

\[
\log(1 - x) = -\sum_{n=1}^{\infty} \frac{x^n}{n}
\]

In your case, where \(x = t\Phi\), this becomes:

\[
\log(1 - t\Phi) = -\sum_{n=1}^{\infty} \frac{(t\Phi)^n}{n}
\]

This series converges for \(|t\Phi| < 1\). Since \(|\Phi| < 1\) as \(\Phi\) is in the unit disc, the series converges for \(|t| < 1/|\Phi|\).\\

Thus, the logarithmic function \(\log(1 - t\Phi)\) can be represented as a sum over the natural numbers as long as the product \(t\Phi\) satisfies the convergence condition.
\item log(1-tΦ) and both have power-series expansions around t = 0
\item exp, log
\item Tr : [V,V] ⭢ ℕ
\item Det : [V,V] ⭢ k
\item 𝕋𝕣 := Σ${}_{i}$ $(-1)^{i}$ * Tr${}^{(-1)^{i}}$(Φ)
\item 𝔻𝕖𝕥 := Π${}_i$ Det${}^{(-1)^{i}}$(Φ)
\item (1 - t * Φ) * adj(1 - t * Φ) = det(1- t * Φ)
\item log(1 - t * Φ) = - ʃ (n : ℕ) (t * Φ)ⁿ
\item is_invertible det( 1 - t * Φ )
\item log( 1/(1-tΦ) ) =  ʃ (n : ℕ) (t * Φ)ⁿ
\item Functional equation given a Poincare duality
\item Sorry, I made a ton of mistakes. I guess I wanted A to be normal in the above. 
\item The power series for log : Mat(n×n) ℂ converges for || Φ - λ I || < 1.
\end{enumerate}


\subsection{L-Functions}

\begin{enumerate}
\item 
\end{enumerate}



\subsection{Etale Homotopy}

Consider one of the internal objects defined in the first repository of $\texttt{ThreeWhiteheadTheoremsandThreePuppeSequences}$:

\begin{enumerate}
\item Internal category
\item Internal groupoid
\item Internal group
\end{enumerate}

The functors satisfy a condition which ensures that smash product and its two analogues be sent to product of ∞-categories, ∞-groupoids, and based connected ∞-groupoids, respectively.\\

\begin{enumerate}
\item How to obtain an ∞-category from a map of A∞-rings
\item How to obtain an ∞-groupoid from a map of E∞-rings
\end{enumerate}


\subsection{Topological E-∞ and A-∞ Rings}

In the third and fourth repositories, we took inspiration from the study of topological frames in thinking about `topological topoi', in which the topological frame arising from the frame with two points is replaced with the internal universe ∞$\_$(∞-Cat). The fourth repository details how three different internal universes ∞$\_$(Directoid), ∞$\_$(Spectroid), and ∞$\_$(Spectrum). Each of these constituents of ∞-Cat is involved in a similar logic to the situation for frames, locales, join lattices, and meet lattices. The third and four repositories, titled $\texttt{GeometricMapsinHigherToposTheory}$ and $\texttt{GeometricMapsinHigherAlgebra}$ detail a way to think of `topological'. A∞-rings and E∞-rings in this way.

It's possible that some of the theorems below could feature nicely in the theory we're striving for:

\begin{enumerate}
\item https://ncatlab.org/nlab/show/compact+Hausdorff+rings+are+profinite
\item 
\end{enumerate}


\subsection{Some Important Mathematical Objects}

\begin{enumerate}
\item GL₁(ℚₚ)-representations and the profinite abelian extensions of ℚₚ
\item GLₙ(ℚₚ)-representations.
\item https://math.dartmouth.edu/~eassaf/Coloquium_talk_-_p-adic_Langlands
\item ℚ((x₁,...,xₙ))
\item We begin the subsection here with the concept of a fixed subobject given an internal action for an object in one of the stable models. We then relate the fixed object to its universal property under certain conditions.
\item Unitary groups
\item One giant group which either (a) quotients onto all unitary groups... (b)
\item ℤ[x₁,...,xₙ] is the derived cohomology ring 
\item I would like to form U(n) and also a group U(∞), the group of unitary operators on Hilbert space
\item BU(n) classifies rank n vector bundles
\item GLₙ(ℂ)?
\item 
\item The classifying space construction is related to the symmetric group and to the symmetric polynoimals
\item ℤ[x₁,...,xₙ] is to do with the cohomology ring of U(n)
\end{enumerate}



\end{document}
