\documentclass{book}
\usepackage{blindtext, hyperref, verbatim, minted, graphicx, amssymb, textcomp, enumerate, tcolorbox, newunicodechar, textgreek, wasysym, tipa, eso-pic, lipsum, bbold, dsfont}
\usepackage[margin=1.3in]{geometry}
\usepackage{longtable}
\usepackage{lmodern} % Add this line to use the Latin Modern font
\renewcommand\familydefault{\sfdefault}
\usepackage{fontspec}
\usepackage{newunicodechar}

\newenvironment{definition}
{%
    \begin{center}
    \textbf{\large Definition}
    \end{center}
    \begin{quote}
}
{%
    \end{quote}
}

\newunicodechar{×}{$\times$}
\newunicodechar{→}{$\rightarrow$}
\newunicodechar{⟨}{$\langle$}
\newunicodechar{⟩}{$\rangle$}
\newunicodechar{↦}{$\mapsto$}
\newunicodechar{∧}{$\wedge$}
\newunicodechar{∨}{$\vee$}
\newunicodechar{∃}{$\exists$}
\newunicodechar{∀}{$\forall$}
\newunicodechar{¬}{$\neg$}

\newunicodechar{ᵃ}{${}^{\texttt{a}}$}
\newunicodechar{ᵇ}{${}^{\texttt{b}}$}
\newunicodechar{ᶜ}{${}^{\texttt{c}}$}
\newunicodechar{ᵈ}{${}^{\texttt{d}}$}
\newunicodechar{ᵉ}{${}^{\texttt{e}}$}
\newunicodechar{ᶠ}{${}^{\texttt{f}}$}
\newunicodechar{ᵍ}{${}^{\texttt{g}}$}
\newunicodechar{ʰ}{${}^{\texttt{h}}$}
\newunicodechar{ⁱ}{${}^{\texttt{i}}$}
\newunicodechar{ʲ}{${}^{\texttt{j}}$}
\newunicodechar{ᵏ}{${}^{\texttt{k}}$}
\newunicodechar{ˡ}{${}^{\texttt{l}}$}
\newunicodechar{ᵐ}{${}^{\texttt{m}}$}
\newunicodechar{ⁿ}{${}^{\texttt{n}}$}
\newunicodechar{ᵒ}{${}^{\texttt{o}}$}
\newunicodechar{ᵖ}{${}^{\texttt{p}}$}
\newunicodechar{ʳ}{${}^{\texttt{r}}$}
\newunicodechar{ˢ}{${}^{\texttt{s}}$}
\newunicodechar{ᵗ}{${}^{\texttt{t}}$}
\newunicodechar{ᵘ}{${}^{\texttt{u}}$}
\newunicodechar{ᵛ}{${}^{\texttt{v}}$}
\newunicodechar{ʷ}{${}^{\texttt{w}}$}
\newunicodechar{ˣ}{${}^{\texttt{x}}$}
\newunicodechar{ʸ}{${}^{\texttt{y}}$}
\newunicodechar{ᶻ}{${}^{\texttt{z}}$}

\newunicodechar{⁰}{${}^{\texttt{0}}$}
\newunicodechar{¹}{${}^{\texttt{1}}$}
\newunicodechar{²}{${}^{\texttt{2}}$}
\newunicodechar{³}{${}^{\texttt{3}}$}
\newunicodechar{⁴}{${}^{\texttt{4}}$}
\newunicodechar{⁵}{${}^{\texttt{5}}$}
\newunicodechar{⁶}{${}^{\texttt{6}}$}
\newunicodechar{⁷}{${}^{\texttt{7}}$}
\newunicodechar{⁸}{${}^{\texttt{8}}$}
\newunicodechar{⁹}{${}^{\texttt{9}}$}

\newunicodechar{⁻}{${}^{\texttt{-}}$}

\newunicodechar{ᵒ}{${}^{\texttt{o}}$}
\newunicodechar{ᵖ}{${}^{\texttt{p}}$}

\newunicodechar{⁻}{${}^{\texttt{-}}$}
\newunicodechar{¹}{${}^{\texttt{1}}$}

\newunicodechar{₀}{${}_{\texttt{0}}$}
\newunicodechar{₁}{${}_{\texttt{1}}$}
\newunicodechar{₂}{${}_{\texttt{2}}$}
\newunicodechar{₃}{${}_{\texttt{3}}$}
\newunicodechar{₄}{${}_{\texttt{4}}$}
\newunicodechar{₅}{${}_{\texttt{5}}$}
\newunicodechar{₆}{${}_{\texttt{6}}$}
\newunicodechar{₇}{${}_{\texttt{7}}$}
\newunicodechar{₈}{${}_{\texttt{8}}$}
\newunicodechar{₉}{${}_{\texttt{9}}$}

\newunicodechar{𝚫}{$\Delta$}

\newunicodechar{ʃ}{$\int$}
\newunicodechar{⊗}{$\otimes$}


\newunicodechar{𝔸}{$\mathbb{A}$}
\newunicodechar{𝔹}{$\mathbb{B}$}
\newunicodechar{ℂ}{$\mathbb{C}$}
%\newunicodechar{}{$\mathbb{D}$}
%\newunicodechar{}{$\mathbb{E}$}
\newunicodechar{𝔽}{$\mathbb{F}$}
\newunicodechar{𝔾}{$\mathbb{G}$}
%\newunicodechar{}{$\mathbb{H}$}
%\newunicodechar{}{$\mathbb{I}$}
%\newunicodechar{}{$\mathbb{J}$}
%\newunicodechar{}{$\mathbb{K}$}
%\newunicodechar{}{$\mathbb{L}$}
%\newunicodechar{}{$\mathbb{M}$}
\newunicodechar{ℕ}{$\mathbb{N}$}
%\newunicodechar{}{$\mathbb{O}$}
\newunicodechar{ℙ}{$\mathbb{P}$}
%\newunicodechar{}{$\mathbb{Q}$}
\newunicodechar{ℝ}{$\mathbb{R}$}
\newunicodechar{𝕊}{$\mathbb{S}$}
%\newunicodechar{}{$\mathbb{T}$}
%\newunicodechar{}{$\mathbb{U}$}
%\newunicodechar{}{$\mathbb{V}$}
%\newunicodechar{}{$\mathbb{W}$}
%\newunicodechar{}{$\mathbb{X}$}
%\newunicodechar{}{$\mathbb{Y}$}
\newunicodechar{ℤ}{$\mathbb{Z}$}
\newunicodechar{𝕒}{$\mathbb{a}$}
\newunicodechar{𝕓}{$\mathbb{b}$}
%\newunicodechar{}{$\mathbb{c}$}
\newunicodechar{𝕕}{$\mathbb{d}$}
\newunicodechar{𝕖}{$\mathbb{e}$}
\newunicodechar{𝕗}{$\mathbb{f}$}
%\newunicodechar{}{$\mathbb{g}$}
%\newunicodechar{}{$\mathbb{h}$}
%\newunicodechar{}{$\mathbb{i}$}
\newunicodechar{𝕛}{$\mathbb{j}$}
%\newunicodechar{}{$\mathbb{k}$}
%\newunicodechar{}{$\mathbb{l}$}
%\newunicodechar{}{$\mathbb{m}$}
%\newunicodechar{}{$\mathbb{n}$}
\newunicodechar{𝕠}{$\mathbb{o}$}
\newunicodechar{𝕡}{$\mathbb{p}$}
%\newunicodechar{}{$\mathbb{q}$}
\newunicodechar{𝕣}{$\mathbb{r}$}
%\newunicodechar{}{$\mathbb{s}$}
\newunicodechar{𝕥}{$\mathbb{t}$}
%\newunicodechar{}{$\mathbb{u}$}
%\newunicodechar{}{$\mathbb{v}$}
%\newunicodechar{}{$\mathbb{w}$}
%\newunicodechar{}{$\mathbb{x}$}
%\newunicodechar{}{$\mathbb{y}$}
%\newunicodechar{}{$\mathbb{z}$}

\usepackage{eucal}
\usepackage{euscript}

\newunicodechar{𝔄}{$\mathfrak{A}$}
\newunicodechar{𝔅}{$\mathfrak{B}$}
\newunicodechar{ℭ}{$\mathfrak{C}$}
\newunicodechar{𝔇}{$\mathfrak{D}$}
\newunicodechar{𝔈}{$\mathfrak{E}$}
\newunicodechar{𝔉}{$\mathfrak{F}$}
\newunicodechar{𝔊}{$\mathfrak{G}$}
\newunicodechar{ℌ}{$\mathfrak{H}$}
\newunicodechar{ℑ}{$\mathfrak{I}$}
\newunicodechar{𝔍}{$\mathfrak{J}$}
\newunicodechar{𝔎}{$\mathfrak{K}$}
\newunicodechar{𝔏}{$\mathfrak{L}$}
\newunicodechar{𝔐}{$\mathfrak{M}$}
\newunicodechar{𝔑}{$\mathfrak{N}$}
\newunicodechar{𝔒}{$\mathfrak{O}$}
\newunicodechar{𝔓}{$\mathfrak{P}$}
\newunicodechar{𝔔}{$\mathfrak{Q}$}
\newunicodechar{ℜ}{$\mathfrak{R}$}
\newunicodechar{𝔖}{$\mathfrak{S}$}
\newunicodechar{𝔗}{$\mathfrak{T}$}
\newunicodechar{𝔘}{$\mathfrak{U}$}
\newunicodechar{𝔙}{$\mathfrak{V}$}
\newunicodechar{𝔚}{$\mathfrak{W}$}
\newunicodechar{𝔛}{$\mathfrak{X}$}
\newunicodechar{𝔜}{$\mathfrak{Y}$}
\newunicodechar{ℨ}{$\mathfrak{Z}$}

\newunicodechar{𝔞}{$\mathfrak{a}$}
\newunicodechar{𝔟}{$\mathfrak b$}
\newunicodechar{𝔠}{$\mathfrak{c}$}
\newunicodechar{𝔡}{$\mathfrak{d}$}
\newunicodechar{𝔢}{$\mathfrak{e}$}
\newunicodechar{𝔣}{$\mathfrak{f}$}
\newunicodechar{𝔤}{$\mathfrak{g}$}
\newunicodechar{𝔥}{$\mathfrak{h}$}
\newunicodechar{𝔦}{$\mathfrak{i}$}
\newunicodechar{𝔧}{$\mathfrak{j}$}
\newunicodechar{𝔨}{$\mathfrak{k}$}
\newunicodechar{𝔩}{$\mathfrak{l}$}
\newunicodechar{𝔪}{$\mathfrak{m}$}
\newunicodechar{𝔫}{$\mathfrak{n}$}
\newunicodechar{𝔬}{$\mathfrak{o}$}
\newunicodechar{𝔭}{$\mathfrak{p}$}
\newunicodechar{𝔮}{$\mathfrak{q}$}
\newunicodechar{𝔯}{$\mathfrak{r}$}
\newunicodechar{𝔰}{$\mathfrak{s}$}
\newunicodechar{𝔱}{$\mathfrak{t}$}
\newunicodechar{𝔲}{$\mathfrak{u}$}
\newunicodechar{𝔳}{$\mathfrak{v}$}
\newunicodechar{𝔴}{$\mathfrak{w}$}
\newunicodechar{𝔵}{$\mathfrak{x}$}
\newunicodechar{𝔶}{$\mathfrak{y}$}
\newunicodechar{𝔷}{$\mathfrak{z}$}



\newunicodechar{𝐂}{${\bf{C}}$}
\newunicodechar{𝐈}{${\bf{I}}$}
\newunicodechar{𝐒}{${\bf{S}}$}

\newunicodechar{𝐚}{${\bf{a}}$}
\newunicodechar{𝐞}{${\bf{e}}$}
\newunicodechar{𝐟}{${\bf{f}}$}
\newunicodechar{𝐧}{${\bf{n}}$}
\newunicodechar{𝐭}{${\bf{t}}$}

\newunicodechar{⊣}{\ensuremath{\dashv}}
\newunicodechar{ॱ}{${}_{\cdot}$}
\newunicodechar{𛲔}{${}^{\cdot}$}
\newunicodechar{⇄}{$\rightleftarrows$}
\newunicodechar{⇆}{$\leftrightarrows$}

\newunicodechar{ꜝ}{$\raisebox{1ex}{\scalebox{0.5}{\texttt{!}}}$}
\newunicodechar{ꜞ}{$\raisebox{1ex}{\scalebox{0.5}{\texttt{¡}}}$}

\newunicodechar{𝟙}{$\mathbb{1}$}
\newunicodechar{∘}{$\circ$}

\newunicodechar{𝟏}{${\bold{1}}$}
\newunicodechar{⭢}{$\longrightarrow$}
\newunicodechar{•}{${\bullet}$}
\newunicodechar{∙}{${\bullet}$}

\newunicodechar{⇉}{$\rightrightarrows$}
\newunicodechar{よ}{$\includegraphics[width=0.27cm,height=0.27cm]{yon.png}$}

\newunicodechar{⊥}{$\bot$}

\newunicodechar{∼}{$\sim$}
\newunicodechar{≃}{$\simeq$}
\newunicodechar{≅}{$\cong$}
\newunicodechar{∞}{$\infty$}

\newunicodechar{α}{$\alpha$}
\newunicodechar{β}{$\beta$}
\newunicodechar{γ}{$\gamma$}
\newunicodechar{δ}{$\delta$}
\newunicodechar{ε}{$\epsilon$}
\newunicodechar{η}{$\eta$}
\newunicodechar{ζ}{$\zeta$}
\newunicodechar{θ}{$\theta$}
\newunicodechar{ι}{$\iota$}
\newunicodechar{μ}{$\mu$}
\newunicodechar{κ}{$\kappa$}
\newunicodechar{λ}{$\lambda$}
\newunicodechar{ρ}{$\rho$}
\newunicodechar{π}{$\pi$}
\newunicodechar{σ}{$\sigma$}
\newunicodechar{τ}{$\tau$}
\newunicodechar{υ}{$\upsilon$}
\newunicodechar{φ}{$\phi$}
\newunicodechar{ψ}{$\psi$}
\newunicodechar{χ}{$\chi$}
\newunicodechar{ω}{$\omega$}



\makeatletter
\newcommand*{\shifttext}[2]{\settowidth{\@tempdima}{#2}\makebox[\@tempdima]{\hspace*{#1}#2}}
\makeatother
\definecolor{Red}{cmyk}{0.1, 0.70, 0.65, 0.00, 1.00}
\definecolor{Blue}{cmyk}{0.9, 0.2, 0.2, 0.00, 1.00}
\definecolor{Yellow}{cmyk}{0.0, 0.00, 0.7, 0.00, 0.5}
\definecolor{Green}{cmyk}{0.6, 0.0, 0.6, 0.00, 1.00}
\definecolor{Purple}{cmyk}{0.8, 0.3, 0.3, 0.00, 1.00}
\definecolor{Orange}{cmyk}{0.0, 0.3, 0.7, 0.00, 1.00}
\definecolor{Grey}{cmyk}{0.13, 0.13, 0.13, 0.00, 1.00}
\newcounter{definitioncounter}
\setcounter{definitioncounter}{1}
\newcounter{theoremcounter}
\setcounter{theoremcounter}{1}
\newcounter{printcounter}
\setcounter{printcounter}{1}
\newcounter{examplecounter}
\setcounter{examplecounter}{1}
\newcounter{ccounter}
\setcounter{ccounter}{1}
\newcounter{pcounter}
\setcounter{pcounter}{1}
\newcounter{lcounter}
\setcounter{lcounter}{1}
\newcounter{sectioncount}
\newcounter{subsectioncount}
\setcounter{sectioncount}{1}
\renewcommand{\section}[1]{\newpage\ \\ \ \\ \begin{center} \scalebox{1.5}{\texttt{\thesectioncount . #1}} \stepcounter{sectioncount} \setcounter{subsectioncount}{1} \end{center} \begin{center} \ \\ \ \\ \thispagestyle{empty} \end{center}}
\renewcommand{\subsection}[1]{\texttt{\thesubsectioncount . #1} \stepcounter{subsectioncount}}
\renewcommand{\backslash}{\reflectbox{\texttt{/}}}



\renewcommand{\chapter}[1]{
\newpage
{
\Huge 
\begin{center}
\ \\
\ \\
\thispagestyle{empty}
\texttt{#1}
\end{center}}
\ \\
\ \\
}

\begin{document}

In the document below I state my interests concerning mathematics in Lean 4. Over the next four year period, I would like to accomplish stages I and II of the plan below. These plans are detailed in 

\section{\scalebox{0.6}{Stage I: twenty-four goals concerning homotopy and stable homotopy}}

In stage one, we would like to detail twenty-four major theorems:

\begin{enumerate}
\item The Whitehead theorem
\item The Puppe sequence and its exactness
\item An analogue of the Whitehead theorem for ∞-groupoids featuring an interval object along with two different structures from internal category theory (see Janelidze and Bourceux chapter 7).
\item An analogue of the Puppe sequence for ∞-groupoids featuring an interval object along with two different structures from internal category theory (see Janelidze and Bourceux chapter 7).
\item 
\end{enumerate}

\begin{enumerate}
\item (Three Whitehead Theorems and Three Puppe Sequences, 2024)
\begin{enumerate}
\item See the goals listed in the pdf file under the $\texttt{Three Whitehead Theorems and Three Puppe Sequences}$ repository.
\item Term length, typeclasses, and inductive types
\end{enumerate}
\item (Three Stable Homotopy Categories and Three Long Exact Sequences, 2025)
\begin{enumerate}
\item See the goals listed in the pdf file under the $\texttt{Three Stable Homotopy Categories and Three Long Exact Sequences}$ repository 
\item Term length, typeclasses, and inductive types
\end{enumerate}
\end{enumerate}



\section{\scalebox{0.6}{Stage II: Goals concerning geometric maps in topos theory and stable homotopy}}

{\bf STAGE II (TOPOLOGY):}
\begin{enumerate}
\item (Topology Project I, 2026)
\begin{enumerate}
\item The construction of the following universe objects:
\begin{enumerate}
\item ∞${}\_$(∞-Cat) : ∞-Cat.α
\item ∞${}\_$(∞-Grpd) : ∞-Cat.α
\item ∞${}\_$(∞-Grpd₀) : ∞-Cat.α
\item D(∞${}\_$(∞-Cat)) : ∞-Cat.α
\item D(∞${}\_$(∞-Grpd)) : ∞-Cat.α
\item D(∞${}\_$(∞-Grpd₀)) : ∞-Cat.α
\end{enumerate}
\item The construction of the following:
\begin{enumerate}
\item ∞${}\_$(∞-Cat)⁄C : ∞-Cat.α
\item ∞${}\_$(∞-Grpd)⁄G : ∞-Cat.α
\item ∞${}\_$(∞-Grpd₀)⁄G₀ : ∞-Cat.α
\item D(∞${}\_$(∞-Cat)⁄C) : ∞-Cat.α
\item D(∞${}\_$(∞-Grpd)⁄G) : ∞-Cat.α
\item D(∞${}\_$(∞-Grpd₀)⁄G₀) : ∞-Cat.α
\end{enumerate}
\item The theory of pointed Kan extensions from the perspective of directed homotopy pullback of ⊥ : * ⭢ ∞${}\_$(∞-Cat)... or perhaps some kind of directed derived category instead. See our "\texttt{pullback}$\_$\texttt{system}" structure.
\item The relationship between the projection formulas and the condition of being an open or closed map in frame-local theory. See \href{https://ncatlab.org/nlab/show/closed+morphism}{here} and \href{https://ncatlab.org/nlab/show/open+morphism}{here}.
\item Defining etale unions and proper intersections in the style of Martin Escardo and Paul Taylor.
\item Proper intersections of closed are closed and etale unions of opens are open.
\item Proper iff universally closed, Étale iff universally open.
\item Proper iff cofiltered limit in the "category of actions", Étale iff filtered colimit in the "category of actions". In the case of frame-local theory, this is a certain structure to do with a frame (a particular join lattice), but viewed in the category of complete partial orders as opposed to frames and locales.
\item Tychonoff's theorem: 
\end{enumerate}
\item (Topology Project II, 2027):
\begin{enumerate}
\item The construction of the following universe objects:
\begin{enumerate}
\item ∞${}\_$(Directoid) : ∞-Cat.α
\item ∞${}\_$(Spectroid) : ∞-Cat.α
\item ∞${}\_$(Spectra) : ∞-Cat.α
\item D(∞${}\_$(Directoid)) : ∞-Cat.α
\item D(∞${}\_$(Spectroid)) : ∞-Cat.α
\item D(∞${}\_$(Spectra)) : ∞-Cat.α
\end{enumerate}
\item The construction of the following universe objects:
\begin{enumerate}
\item Mon D(∞${}\_$(Directoid)) : ∞-Cat.α
\item Mon D(∞${}\_$(Spectroid)) : ∞-Cat.α
\item Mon D(∞${}\_$(Spectra)) : ∞-Cat.α
\item Com D(∞${}\_$(Directoid)) : ∞-Cat.α
\item Com D(∞${}\_$(Spectroid)) : ∞-Cat.α
\item Com D(∞${}\_$(Spectra)) : ∞-Cat.α
\end{enumerate}
\item The construction of the following:
\begin{enumerate}
\item Act D : ∞-Cat.α 
\item Act S : ∞-Cat.α
\item Act S : ∞-Cat.α
\end{enumerate}
\item Ramification 
\item The closed and open conditions and their relationship with the projection formulas.
\item Universally closed intersections of closed maps are closed and universally open etale unions of opens are open.
\item Showing that proper is equivalent to being universally closed with universally closed diagonal, and that Étale is equivalent to being universally open with universally open diagonal. This much can be shown first for the case of partial orders, and then for the case of an unstraightening system.
\item Proper iff cofiltered limit in the category of actions, Étale iff filtered colimit in the category of actions.
\item Tychonoff's theorem.
\item What is "Conglomeration Index"?
\item Ramified extensions, Unseparated extensions.
\end{enumerate}
\end{enumerate}


\section{STAGE III: Number theory and beyond}

STAGE III:
\begin{enumerate}
\item Exact couples
\item Spectral sequences
\item The main properties of 𝕋𝕣 := Σ${}_{i}$ $(-1)^{i}$ * Tr${}^{(-1)^{i}}$(Φ)
\item The main properties of 𝔻𝕖𝕥 := Π${}_i$ Det${}^{(-1)^{i}}$(Φ)
\item 1/(1-z) : 𝔻 in the context of L-functions
\item 1/(1-z) : ??? is to do with the homotopy coequilizer of 1 : ℂ ⭢ ℂ and z: ℂ ⭢ ℂ namely Δ¹ × ℂ...into ℂ ... 
\item exp : ℂ ⭢ ℂˣ
\item log : π⃡₀.obj (P⃡.obj ℂˣ) ⭢ ℂ 
\item 1/(1-z)
\item 𝔻
\item The main features of 
\item Fox's theorem in my ETCC
\item Cohomology for E${}^{∞}$-Rings, E⃡${}^{∞}$-Rings, and E⃗${}^{∞}$-Rings
\item Cohomology for E${}^{∞}$-Modules, E⃡${}^{∞}$-Modules, and E⃗${}^{∞}$-Modules
\item Flat connections as elements of the ℝ-linear dual of $\texttt{Ω\_(Spectra)C}$, where C is the cohomology content (which is an internal monoid)
\item P.S. there is also the commutative C(Y)-algebra C(X) ... Ω${}^{∞}$, Ω⃡${}^{∞}$, and Ω⃗${}^{∞}$
\item P.S. there is also the C(Y)-algebra C(X).
\item Poincare duality
\item "smooth" as "etale locally free"
\item The Cayley-Hamilton Theorem for objects with Poincare duality
\item The adjunct theorem for Poincare dual objects
\item The Kunneth theorem
\item Well behaved cycle maps between pullback and cohomology classes
\item Pullback and cup product
\item The Lefschetz Fixed Point Theorem
\item The Idelic Fourier Transform
\item The extended Poisson summation formula from Tate's thesis
\item Functional equation given Poincare duality
\item Algebraic nature of the roots at times
\item Cohomology with rational coefficients
\item p-adic cohomology in the case of a self dual field with self dual functional
\item How to obtain a CW-complex from a map of E∞-rings.
\item Borel-Moore homology and cohomology with compact support
\begin{enumerate}
\item Homology, p𛲔 pॱ 𝟙
\item Cohomology Cmp(p)𛲔 pॱ 𝟙
\item Borel-Moore Cohomology Cmp(p)ॱ Cmp(p)𛲔
\item Cohomology with compact support pॱ Cmp(p)𛲔
\item https://homepages.warwick.ac.uk/staff/Martin.Gallauer/docs/m6ff.pdf
\end{enumerate}
\item Such duality results as:
\begin{enumerate}
\item Serre duality
\item Grothendieck local duality
\item Matlis duality
\item Coherent duality
\item Verdier duality
\item Poincare duality
\end{enumerate}
\end{enumerate}

\end{document}
